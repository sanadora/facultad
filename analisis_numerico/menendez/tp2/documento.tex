\documentclass{article}

\usepackage{amsmath}
\usepackage{graphicx}
\usepackage{float}
\usepackage{hyperref}
\usepackage{booktabs}

\title{Trabajo Practico Nro 1}
\date{10-2014}
\author{Gortari, Bautista - Mauro, Pedro}

\begin{document}

\pagenumbering{gobble}
%\maketitle
\begin{center}
\large{Facultad de Ingenieria | Universidad de Buenos Aires}
\newline
\newline
\large{2do Cuatrimestre | 2014}
\newline
\newline

\huge{\textbf{75.12 | Analisis Numerico IA}}
\newline
\newline
\huge{\textbf{Trabajo Practico \#2}}
\newline
\newline
\large{\textbf{Distribucion de temperaturas en una barra de metal}}
\newline
\newline
\end{center}
\begin{table}[h!]
  \begin{center}
    \label{tab:table1}
    \begin{tabular}{ccc}
      \toprule
      Grupo Numero &11 \\
      \midrule
      Bautista Gortari &93733\\
      \midrule
      Mauro Pedro &93664 \\
      \bottomrule
    \end{tabular}
  \end{center}
\end{table}
\newpage
\tableofcontents
\newpage
\pagenumbering{arabic}



\section{Sistema de ecuaciones lineales a resolver}
\begin{align*}
u_1^2 &= 0\\
-r.u_1^2 + (2+2r).u_2^2 - r.u_3^2 &= r.u_1^1 + (2-2r).u_2^1 + r.u_3^1\\
&\vdots\\
-r.u_{n-2}^2 + (2+2r).u_{n-1}^2 - r.u_{n}^2 &= r.u_{n-2}^1 + (2-2r).u_{n-1}^1 + r.u_{n}^1\\
u_n^2&=0\\
\end{align*}

Este seria el sistema de ecuaciones a resolver que representan el paso del tiempo
inicial (1) a uno posterior (2). Luego, si uno quisiera tener la
evolucion temporal del sistema, tiene que iterar las veces que sean
necesarias, utilizando el resultado del sistema anterior.

\subsection{Programa Principal}
\begin{verbatim}
# variables iniciales
t0 = clock();
longitud = 20; # [centimetros]
tiempo = 60; # [segundos]
k = 0.120;
c = 0.114;
p = 7.8;
mu = k / (c * p);
dt = 0.1;
dx = 0.5;
r = (mu * dt) / (dx ** 2)
# m: cantidad de puntos
m = (longitud / dx) + 1
# n: cant de ptos importantes para la matriz
n = m - 2;

A = zeros(n,n);
x = zeros(m,1);
b = zeros(m,1);

# se carga la matriz A, con los coeficientes de las ecuaciones
A(1,1) = 2 + 2*r;
A(1,2) = -r;
for j = 2:n-1;
  A(j,j) = 2 + 2*r;
  A(j,j-1) = -r;
  A(j,j+1) = -r;
endfor;
A(n,n-1) = -r;
A(n,n)   = 2 + 2*r;

for j = 0:m-1;
  if (j*dx) <= 10;
    x(j+1,1) = 10 * j*dx;
  else
    x(j+1,1) = 200 - 10 * j*dx;
  endif;
endfor;


disp("1) Eliminacion Gausseana");
disp("2) Doolittle");
disp("3) Cholesky");
disp("4) Gauss-Seidel");
disp("--------------------------------");
switch menu = input("Que metodo quiere utilizar? [1-4]  > ");
  case 2
    [l, u, p] = lu(A);
  case 3
    L=chol(A,'lower');
end;

for i = 1: tiempo / dt;
  disp(i);
  for j = 2: m-1;
    b(j,1) = r * x(j-1,1) + (2-2*r) * x(j,1) + r * x(j+1,1);
  endfor;

  switch menu;
         case 1
           x(2:m-1 , 1) = eliminacion_gaussiana(A,b(2:m-1 , 1),false);
         case 2
           y = eliminacion_l(l,b(2:m-1,1));
           x(2:m-1 , 1) = eliminacion_gaussiana(u,y,true);
         case 3
           y = eliminacion_l(L,b(2:m-1,1));
           x(2:m-1 , 1) = eliminacion_gaussiana(transpose(L),y,true);
         case 4
           error = 99;
           while error > 0.001
             x_ant = x(2:m-1 , 1);
             x(2:m-1 , 1) = gauss_seidel(A,b(2:m-1 , 1), x(2:m-1 , 1));
             x_nuevo = x(2:m-1 , 1);
             error = max( abs( x_nuevo - x_ant));
           endwhile;


  end;
endfor;

disp("RESULTADO");
disp(x);

printf("Tiempo transcurrido: %f segundos.\n",etime(clock(),t0));
\end{verbatim}


\subsection{Eliminacion Gaussiana}
\begin{verbatim}
function [ x ] = eliminacion_gaussiana( A, b , triangular)
%Algoritmo eliminacion gaussiana
%A= input ('matriz cuadrada A: ')
%b= input ('vector de terminos independientes b: ')
[n, q]=size(A);
if !triangular
  for i=1:n-1;
    for k=i+1:n ;
      m=A(k,i)/A(i,i);
      for j= i+1:n;
        A(k,j)= A(k,j)-m*A(i,j);
      end
      A(k,i)=0;
      b(k)=b(k)-m*b(i);
    end
  end
endif;

x=zeros(n,1);
#disp(A);
#disp(b);
for i=n:-1:1;
    aux=0;
    for j=i+1:n
        aux=aux+A(i,j)*x(j);
    end
    x(i)=(b(i)-aux)/A(i,i);
end
\end{verbatim}

\subsection{Gauss-Seidel}
\begin{verbatim}
function x_new = gauss_seidel ( a, b, x )
  [n q] = size(a);
  x_new = zeros ( n, 1 );

  for i = 1 : n
    x_new(i) = b(i);
    x_new(i) = x_new(i) - a(i,1:i-1) * x_new(1:i-1);
    x_new(i) = x_new(i) - a(i,i+1:n) * x(i+1:n);
    x_new(i) = x_new(i) / a(i,i);
  end

  return
end
\end{verbatim}

\subsection{Comentarios}
Para resolver el problema, elegimos $\Delta x= 0.5 cm$ 
y $\Delta t = 0.01s$, que satisface $ r<1$. Elegimos esos
valores, principalmente porque vimos que resultaban un buen compromiso
entre el tiempo de resolucion del problema, y la precision del
resultado. Ademas, consideramos que son intervalos de espacio y tiempo
suficientes para representar correctamente las dimensiones del
problema. 
Inicialmente tuvimos problemas que se originaban en la eleccion de
estos valores. Lo que ocurrio fue que, sin haber analizado la
importancia de tener un r menor a 1, tomamos ciertos valores de dx y dy, efectuamos los calculos,
y obtuvimos resultados sin sentido.



El metodo mas eficiente resulto ser Gauss-Seidel. Esto fue lo
previsto, dado que la matriz de coeficientes resulta tener muy
dispersa, lo que afecta al rendimiento de los metodos directos
(Cholesky, Doolittle y Eliminacion Gaussiana) que funcionan mejor con
matrices densas. Por el contrario, Gauss-Seidel, al ser un metodo
iterativo, se ve beneficiado por esto.

Vale alcarar que al principio tuvimos inconvenientes con los metodos,
ya que los mejores tiempos resultaban ser mediante Eliminacion
Gausseana, contrario a lo esperado.
Luego de revisar los algoritmos nos dimos cuenta que estabamos
haciendo operaciones innecesarias sobre las matrices, desaprovechando
justamente las ventajas de estos metodos. Haciendo las modificaciones
pertinentes, pudimos reducir el costo computacional en gran medida, y
ver la superioridad del metodo de Gauss-Seidel en primer lugar, y
Choleksy y Doolittle en un segundo plano.

\section{Graficos espacio-temporales}

\begin{figure}[H]
  \includegraphics[width=0.7\textwidth]{evol-espacial.png}
  \centering
  \caption{Diagrama de temperaturas para 3 tiempos distintos.}
  \label{fig:espacial}
\end{figure}

En la figura \ref{fig:espacial} se puede ver como la temperatura se
dicipa solo por los extremos de la barra, y estos actuan como una
suerte de ``sumideros''.

\begin{figure}[H]
  \includegraphics[width=0.7\textwidth]{evol-temporal.png}
  \centering
  \caption{Temperatura en funcion del tiempo, para 3 puntos de la barra.}
  \label{fig:temporal}
\end{figure}
La figura \ref{fig:temporal} muestra como las funciones de temperatura
en los 3 puntos elegidos convergen al 0, que es la temperatura de equilibrio.

\section{Costo Computacional y Metodo mas eficiente}
Para calcular el costo computacional al resolver los algoritmos, nos
basamos solamente en el costo temporal, ignorando el espacial. Para
ello, usamos la funcion de octave etime (``elapsed time'') y clock,
las cuales se utilizan al principio y al final del programa para
contar directamente la diferencia entre las dos instrucciones de
clock().
Los resultados fueron los siguientes: 
Eliminacion Gausseana: 1267,04 segundos.
Cholesky: 65,14 segundos.
Doolittle: 62,16 segundos.
\textbf{Gauss-Seidel: 28,48 segundos.}
\newline
Como se puede observar, el metodo mas eficiente es el de Gauss-Seidel,
dadas las caracteristicas de la matriz de coeficientes para el
problema dado.
\section{Analisis de la sensibilidad}
\subsection{Con respecto a dt}
Vimos que variando el dt, desde dt=1 hasta dt=0.01, no se aprecian
cambios significativos en el vector resultado. Recien en el cuarto
decimal, se presentaban las diferencias. Esto nos hizo ver que
inicialmente podriamos haber tomado un dt mayor al elegido, y asi
reducir el costo computacional de las pruebas, y obtener resultados
mas rapidamente, casi sin alteraciones.

\subsection{Con respecto a dx}
Cambiando los valores de dx, desde dx=0.5 a dx=1 primero, y luego a
dx=1, se observan cambios notables en la respuesta. Cambiar el dx, en
primera instancia, implica variar el tamanio del vector resultado,
aunque igualmente se pueden comparar los puntos homologos al dx
original y comparar sus valores. Sobre estos valores es que notamos
una diferencia significativa.

\subsection{Con respecto al material}
Como era de esperarse, el material tiene una gran influencia en el
resultado, para el final del proceso. Las temperaturas en todos los
puntos, difieren ampliamente. Esto tiene que ver con las propiedades
de los materiales analizados. 
Primero, comparamos con Aluminio, el cual tenia un pico de
temperaturas (en el centro de la barra), de 23 grados, mucho menor a
los 68 grados de la barra de Acero. Luego, haciendo lo mismo con la
barra de Plata, resulto una temperatura menor en el mismo punto, de 7
grados.

Si bien la mayor sensibilidad es con respecto al material, esto no nos
preocupa, ya que es algo natural. Nuesto estudio deberia centrarse en
las variables dt y dx, cuya eleccion hace que varie (en mayor o menor
medida) el resultado, que debiera ser siempre el mismo, ya que la
naturaleza del problema no se altera, sino su modelado. Una condicion
necesaria que tiene que cumplir la eleccion de dx y dt es que el
$r$ resulte ser menor a 1. Con esto nos aseguramos que el
resultado represente de manera certera la realidad del problema.

\section{Equilibrio en el Sistema}
Debido a las condiciones establecidas para la barra, esto es, que la
temperatura en los extremos es siempre cero, concluimos lo siguiente:
sabemos que el flujo de calor en la barra es proporcional al gradiente de
temperaturas dentro de la misma y, por lo tanto, el equilibrio se
alcanza cuando todos los puntos de la misma esten a igual temperatura;
como por la condicion de borde impuesta, la temperatura en ambos
extremos es de cero grados Celcius, este va a ser el estado final para
todos los puntos y se alcanzara con el correr del tiempo.


\newpage
\begin{appendix}
  \listoffigures
\end{appendix}

\end{document}
