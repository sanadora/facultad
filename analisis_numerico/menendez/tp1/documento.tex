\documentclass{article}

\usepackage{amsmath}
\usepackage{graphicx}
\usepackage{float}
\usepackage{hyperref}

\title{Trabajo Practico Nro 1}
\date{09-2014}
\author{Gortari, Bautista - Mauro, Pedro}

\begin{document}


\pagenumbering{gobble}
\maketitle
\newpage
\tableofcontents
\newpage
\pagenumbering{arabic}

\section{Investigacion}
El problema elegido es la \textbf{Simulacion Numerica de la combustion en una
caldera a gas natural}. La solucion fue abordada por el metodo de los
elementos finitos, que es un metodo numerico general para la
aproximacion de soluciones de ecuaciones diferenciales parciales muy
utilizado en diversos problemas fisicos e ingenieriles.

La combustion de hidrocarburos en aire es gobernada por mecanismos
multipasos que comprenden cien o mas reacciones, que si bien no son
imposibles de resolver, se termina obscureciendo el progreso de las
reacciones y se dificulta ganar una vision interna de la naturaleza de
la combustion. Para casos practicos, es conveniente reducir el numero
de pasos a un pequenio numero de reacciones significativas.

La tarea de investigacion consistia, entonces, en la simulacion numerica de una
llama y su comparacion con datos experimentales. Obtener los datos en
una gran caldera en servicio no es sencillo ni economico, por lo que
se dispone de un limitado conjunto de datos experimentales. 

El procedimiento de solucion numerica consiste en resolver un conjunto
de ecuaciones utilizando la tecnica de volumenes finitos. En este
procedimiento, las ecuaciones discretizadas son resueltas usando el
esquema diferencial hibrido y una ecuacion de correccion de presion.
El metodo de solucion es desacoplado: las ecuaciones se van
resolviendo de a grupos con un orden establecido, hasta completar la iteracion.
Como para ciertas ecuaciones no se dispone de un modelo adecuado, se
toma la medida de simplificarlas o incluso determinar valores
constantes.

Luego de la explotacion del modelo, los resultados que se obtienen son
considerados satisfactorios, aunque indican la necesidad de seguir refinando los modelos para reproducir
mejor las condiciones reales, aunque la transmision de calor y la composicion
de los gases de escape resultan estar dentro de los valores esperados segun la
experiencia de operacion.

Articulo principal:
\url{http://ojs.c3sl.ufpr.br/ojs/index.php/reterm/article/viewFile/3502/2763}

\section{Resolucion de Problemas}

\subsection{Problema 1 - Numeros Abundantes}
El algoritmo que desarrollamos para resolver lo pedido es el siguiente:
\begin{verbatim}
cr = 0;
i = 1;
do
  s = 0;
  n = (2 * i) + 1;
  for div = 1:(n-1);
      if (rem(n,div) == 0);
          s = s + div;
      endif;
  endfor;
  if ( s > n );
     disp(n);
     cr = cr + 1;
  endif;
  i = i + 1;
until (cr == 5)
\end{verbatim}
Con este algoritmo, los numeros abundantes(impares) encontrados
fueron: 945, 1575, 2205, 2835 y 3465.

\subsection{Problema 2 - Calculo de Mapa Digital de Terreno}
\subsubsection{Datos Disponibles}
\begin{tabular}{l c r}
Coordenada X &Coordenada Y &Altura\\
91	&6	&1.75 \\
157	&155	&1.81\\
100	&279	&2.01\\
208	&279	&1.96\\
104	&388	&2.45\\
305	&0	&2.08\\
330	&108	&2.07\\
372	&204	&2.05\\
384	&282	&2.48\\
422	&390	&2.60\\
510	&10	&1.97\\
532	&93	&1.60\\
614	&390	&2.42\\
608	&12	&1.92\\
712	&117	&1.95\\
718	&385	&2.39\\
822	&52	&1.50\\
816	&248	&2.05\\
923	&5	&1.81\\
920	&173	&1.05\\
923	&385	&2.28\\
1023	&87	&1.60\\
1145	&3	&2.21\\
1105	&380	&2.42\\
\end{tabular}


\begin{figure}[H]
  \includegraphics[width=0.7\textwidth]{imagenes/datos-puntos.png}
  \centering
  \caption{Alturas medidas del terreno.}
  \label{fig:datos-2d}
\end{figure}

La figura \ref{fig:datos-2d} muestra el grafico correspondiente a los
datos disponibles de las alturas del barrio La Ribera.

\subsubsection{Programa Principal}
\begin{verbatim}
% tabla con los valores de altura para el barrio La Ribera
load datos.txt;
% largo y ancho del terreno 
lt = 1150;
at = 400;
% lado del cuadrado que tomamos como celda, en metros 
global lado_celda = 20;

ancho = at / lado_celda;
largo = lt / lado_celda;
terreno = zeros( ancho , largo);

for i = 1:ancho;
    disp(i);
    for j = 1:largo;
        terreno(i,j) = altura_con_IDW(i,j,datos);
    endfor;
endfor;
save resultado.txt;
\end{verbatim}
Este es el codigo correspondiente al programa principal, con el cual
llamamos a las funciones que calculan el terreno en base a distintos
algoritmos. 

\subsubsection{NN - Nearest Neighbour}
\paragraph{Vecino mas cercano} asigna a cada celda el valor del punto
observado mas cercano.
\begin{verbatim}
  function altura = altura_del_NN(i,j,global)
  global lado_celda
  altura = 0;
  distancia  = 9999;
  punto1 = [i*lado_celda , j*lado_celda];
  punto2 = zeros(1,2);
  for k=1:24;
    punto2 = [datos(k,1) , datos(k,2)];
    if (norm(punto1 - punto2) < distancia);
      distancia = norm(punto1 - punto2);
      altura = datos(k,3);
    endif;
  endfor;
  endfunction;
\end{verbatim}

\begin{figure}[H]
  \centering
  \includegraphics[width=0.7\textwidth]{imagenes/NN-05metros.png}
  \caption{NN - 05 metros.}
  \label{fig:mdt-nn05}
\end{figure}
\begin{figure}[h!]
  \centering
  \includegraphics[width=0.7\textwidth]{imagenes/NN-20metros.png}
  \caption{NN - 20 metros.}
  \label{fig:mdt-nn20}
\end{figure}
Las figuras \ref{fig:mdt-nn05} y \ref{fig:mdt-nn20} corresponden a los
MDT con mediante Nearest Neighbour tomando celdas de 05 y 20 metros respectivamente. 




\subsubsection{LI - Linear Interpolation}
\paragraph{Interpolacion Lineal} asigna a la celda el valor del promedio de los k
puntos observados mas cercanos.
\begin{verbatim}
function altura = altura_con_LI(i,j,datos)
% INTERPOLACION LINEAL: asigna a la celda el valor promedio de los k
% puntos observados mas cercanos.
  global lado_celda;
  altura = 0;
  k = 4;
  punto_actual = [i*lado_celda , j*lado_celda];
% espando la tabla de datos con una columna para poner las distancias
  puntos = horzcat(datos,zeros(24,1));
  for i=1:24;
      puntos(i,4) = dist(punto_actual , [datos(i,1) , datos(i,2)]);
  endfor;
% la tabla queda ordenada de menor a mayor por la distancia
  puntos = sortrows(puntos,4);
  for i=1:k;
      altura += puntos(i,3);
  endfor
  altura = altura / k;
endfunction;

function norma = dist(punto_a, punto_b)
  norma = norm(punto_a - punto_b);
endfunction;
\end{verbatim}

Las imagenes
\ref{fig:mdt-nnk305}, \ref{fig:mdt-nnk320}, \ref{fig:mdt-nnk405} y
\ref{fig:mdt-nnk420} corresponden a los MDT mediante Linear
Interpolation para k=3/4 y 05/20 metros de celda, respectivamente.

\begin{figure}[H]
  \centering
  \includegraphics[width=0.7\textwidth]{imagenes/LI-k3-05metros.png}
  \caption{LI - k=3 - 05 metros.}
  \label{fig:mdt-nnk305}
\end{figure}
\begin{figure}
  \centering
  \includegraphics[width=0.7\textwidth]{imagenes/LI-k3-20metros.png}
  \caption{LI - k=3 - 20 metros.}
  \label{fig:mdt-nnk320}
\end{figure}
\begin{figure}[H]
  \centering
  \includegraphics[width=0.7\textwidth]{imagenes/LI-k4-05metros.png}
  \caption{LI - k=4 - 05 metros.}
  \label{fig:mdt-nnk405}
\end{figure}
\begin{figure}
  \centering
  \includegraphics[width=0.7\textwidth]{imagenes/LI-k4-20metros.png}
  \caption{LI - k=4 - 20 metros.}
  \label{fig:mdt-nnk420}
\end{figure}





\subsubsection{IDW}
\paragraph{Inversa distancia ponderada} asigna a la celda la media ponderada
utilizando como factor de ponderacion la inversa de la distancia
elevada al exponente \textit{b}. $z_i$ es el dato observado, $z_j$ es el dato
interpolado y $d_{ij}$ la distancia entre ambos.
\begin{equation*}
z_j = \frac{\sum \frac{z_i}{{d_{ij}}^2}}{\sum \frac{1}{{d_{ij}}^2}}
\end{equation*}
\begin{verbatim}
function altura = altura_con_IDW(i,j,datos)
% INVERSA DISTANCIA PONDERADA asigna a la celda la media ponderada
% utilizando como factor de ponderacion la inversa de la distancia
% elevada al exponente b.
  global lado_celda;
  altura = 0;
  distancia = 0;
  b = 2;
  punto_actual = [i*lado_celda , j*lado_celda];
% expando la tabla de datos con dos columnas para poner, en la primera
% el valor 1/di**b y en la segunda zi/dij**b
  puntos = horzcat(datos,zeros(24,2));
  for i=1:24;
      distancia = dist(punto_actual , [datos(i,1) , datos(i,2)]);
      puntos(i,4) = 1 / (distancia ** b);
      puntos(i,5) = datos(i,3) / (distancia ** b);
  endfor;
  altura = sum(puntos(:,5)) / sum(puntos(:,4));
endfunction;

function norma = dist(punto_a, punto_b)
  norma = norm(punto_a - punto_b);
endfunction;
\end{verbatim}

Los siguientes son los MDTs calculados con el algoritmo anterior. Las
figuras \ref{fig:mdt-idwb205} y \ref{fig:mdt-idwb220} corresponden al
valor de b = 2, con celdas de lado 5 y 20 metros, respectivamente.
Las figuras \ref{fig:mdt-idwb305} y \ref{fig:mdt-idwb320} representan
el calculo con b = 3, para celdas de 5 y 20 metros de lado.

\begin{figure}[H]
  \centering
  \includegraphics[width=0.7\textwidth]{imagenes/IDW-b2-05metros.png}
  \caption{IDW - b=2 - 05 metros.}
  \label{fig:mdt-idwb205}
\end{figure}
\begin{figure}[H]
  \centering
  \includegraphics[width=0.7\textwidth]{imagenes/IDW-b2-20metros.png}
  \caption{IDW - b=2 - 20 metros.}
  \label{fig:mdt-idwb220}
\end{figure}
\begin{figure}[H]
  \centering
  \includegraphics[width=0.7\textwidth]{imagenes/IDW-b3-05metros.png}
  \caption{IDW - b=3 - 05 metros.}
  \label{fig:mdt-idwb305}
\end{figure}
\begin{figure}[H]
  \centering
  \includegraphics[width=0.7\textwidth]{imagenes/IDW-b3-20metros.png}
  \caption{IDW - b=3 - 20 metros.}
  \label{fig:mdt-idwb320}
\end{figure}






\subsubsection{OCTAVE}
\paragraph{griddata} se utiliza para interpolar un conjunto arbitrario
de datos dispersos. GNU Octave utiliza funciones basadas en la
Triangulacion de Delaunay en un plano. Una de ellas es griddata que
genera una malla 2D regular a partir de una serie de datos irregular.
Utiliza tres metodos de interpolacion: nearest, cubic o lineal.
\begin{verbatim}
function altura_con_GRIDDATA(datos)
  x = datos(:,1);
  y = datos(:,2);
  z = datos(:,3);
  [X,Y] = meshgrid(unique(x),unique(y));
  Z = griddata(x,y,z,X,Y);
  W = horzcat(X,Y,Z);
  mesh(Z);
\end{verbatim}
\begin{figure}[H]
  \centering
  \includegraphics[width=0.7\textwidth]{imagenes/GRIDDATA.png}
  \caption{MDT con el metodo GRIDDATA de Octave.}
  \label{fig:mdt-nn}
\end{figure}





\subsubsection{Error Cuadratico Medio y MDT mas representativo}
Consideramos el metodo IDW (con b=2) como el mas indicado para obtener el MDT
mas representativo. Vimos que representaba de manera mas suave las
irregularidades del terreno, a diferencia de los demas.
Ademas, este metodo es el unico que tiene en cuenta la totalidad de
los puntos de los cuales tenemos informacion. Por el contrario, los
metodos NN y LI "desperdician" parte de ella.
Nos parecio importante ponderar la distancia de cada punto
estimado a los datos, ya que no es tan relevante un punto lejano en
comparacion con uno cercano.
No tuvimos en cuenta, para la eleccion de nuestro mejor modelo, el
Error Cuadratico Medio, porque notamos que solo tiene en cuenta el
resultado de pocos puntos, dandoles la libertad a los demas de tener
cualquier valor, sin influir en el error.

Vale aclarar que era de esperar que el error del metodo NN diera cero,
debido a que cada punto evaluado tenia la informacion directamente
tomada de los datos. Sin embargo no fue asi. Esto se debe a la forma
en que dividimos el terreno en una grilla dentro del algoritmo,
tomando a la esquina de una celda como parametro de representacion.
\begin{verbatim}
function resultado = dif_cuadrados_media(datos,terreno,lado_celda)
  diferencia = 0;
  resultado = 0;
  for i=1:24;
    n = floor(datos(i,2) / lado_celda) ;
    m = floor(datos(i,1) / lado_celda) ;
    if n >= size(terreno)(1);
       n -= 1;
    endif
    if n < 1;
       n += 1;
    endif;
    if m >= size(terreno)(2);
       m -= 1;
    endif
    disp(terreno(n,m));
    disp(datos(i,3));
    diferencia = datos(i,3) - terreno(n,m)
    resultado += diferencia ** 2;
  endfor
  resultado = resultado / 24;
endfunction;
\end{verbatim}
Con el algoritmo anterior, resulta que ECM = 0.12161 m para el calculo
mediante IDW (con b=2), y la celda de 20 metros de lado. Para los
metodos de NN y LI (k=3), los resultados son 0.26426 y 0.15611,
respectivamente (ambos con lado = 20 metros).

\subsubsection{Cortes longitudinales y transversales}
Las siguientes imagenes corresponden a los cortes longitudinales y
transversales del MDT mediante IDW con b=2 y celda de 20 metros de
lado.
El terreno tiene una medida de 1150m x 400m. Los primeros dos
graficos, \ref{fig:trans200} y \ref{fig:trans400}, son cortes a lo
largo de los 1150 metros. 
\begin{figure}[H]
  \centering
  \includegraphics[width=0.7\textwidth]{imagenes/fila05.png}
  \caption{Corte longitudinal a los 100m.}
  \label{fig:long100}
\end{figure}
\begin{figure}[H]
  \centering
  \includegraphics[width=0.7\textwidth]{imagenes/fila15.png}
  \caption{Corte longitudinal a los 300m.}
  \label{fig:long300}
\end{figure}

Estos dos ultimos graficos, en cambio, son los cortes a lo ancho de
los 400 metros.
\begin{figure}[H]
  \centering
  \includegraphics[width=0.7\textwidth]{imagenes/columna10.png}
  \caption{Corte transversal a los 200m.}
  \label{fig:trans200}
\end{figure}
\begin{figure}[H]
  \centering
  \includegraphics[width=0.7\textwidth]{imagenes/columna20.png}
  \caption{Corte transversal a los 400m.}
  \label{fig:trans400}
\end{figure}



\subsubsection{Curvas de Nivel}
El software toma los extremos de la funcion, y tantos valores
intermedios hasta llegar al valor que uno le solicita mediante un
numero ingresado como parametro. Estos valores los utiliza para
determinar que curvas de nivel graficar. En la figura \ref{fig:curvas}
utilizamos un n=10, por lo que grafica 10 curvas de nivel, indicando
asi el terreno que esta a la misma altura.
\begin{figure}[H]
  \centering
  \includegraphics[width=0.7\textwidth]{imagenes/curvas-10.png}
  \caption{Curvas de nivel.}
  \label{fig:curvas}
\end{figure}

\subsection{Conclusiones}
Sabemos que estamos trabajando con un modelo del terreno, del cual
tenemos pocos datos (en relacion a la dimension del mismo); y como cada
metodo le da mas importancia a distintos aspectos de los datos, los
resultados difieren bastante entre ellos.
Si a priori conocieramos el aspecto del terreno podriamos tener
preferencia
por algun metodo en particular, que sea consistente con su forma. 

Con respecto a la validacion de los modelos, una manera podria resultar de
comparar (mediante la diferencia de cuadrados media)
nuestros resultados, con un nuevo conjunto de datos tomados del mismo
terreno, para ver si el que elegimos es el que arroja menor diferencia. 
Otra forma de validacion que no requiere una muestra adicional
resulta de ir sacando, de a uno, datos del
primer set y corriendo el modelo con 23 datos. Luego se utiliza el
dato restante y se calcula la diferencia con el estimado
originalmente, para esa posicion. Repitiendo este proceso con todos
los datos y sumando
estas diferencias se tiene un buen parametro de la presicion del MDT.

\newpage
\begin{appendix}
  \listoffigures
\end{appendix}

\end{document}
